\documentclass[10pt]{article}

\usepackage{amsmath}
\usepackage{amssymb}

\title{Manual for nemo and mp2 programs in MADNESS}

\begin{document}

\maketitle{}

\section{Intro}
The nemo program computes molecular HF and DFT energies and gradients using a nuclear correlation factor (ncf) as well as vibrational frequencies.\cite{Bischoff:2014du} 
In contrast, the moldft program does the same without using an ncf.

The mp2 program computes the MP2 energies on top of either the nemo or the moldft HF reference wave function.\cite{Bischoff:2014ky,Bischoff:2013cx}

\section{Input sections}
\subsection{nemo}
The nemo program uses the same input parameters as moldft, with only a few restrictions and extensions.
In addition to the regular moldft input parameters there are the following options (all are optional)
\begin{description}
\item[nuclear\_corrfac:]	[\textit{slater x, gaussslater, polynomialN, none}]\\
with $x>1$ recommended $x=2.0$, and  $3<N<11$, recommended $N=8$. Default is "{}none"{}.
\item[hessian] if present computes the hessian matrix and vibrational frequencies
\item[read\_cphf] if present read in the solutions to the CPHF equations from a prior hessian calculation
\end{description}
There is no open-shell/spin-polarized version of nemo available.

Note there is a inconsistency with the \texttt{eprec} keyword in the geometry section. 
Because there is no singular potential in the nemo calculation, the \texttt{eprec} keyword is used to smoothen the derivatives of the nuclear correlation factor (It is a step function).
Typically a factor of 1e-3 to 1.e-4 will give you micro-Hartree accuracy with respect to the energy of the (regularized) nuclear potential, and about 1 cm-1 accuracy with respect to vibrational frequencies.

\subsection{mp2}
On top of the HF reference wave function (from nemo or moldft) a MP2 calculation may be performed. 
The relevant input is contained in the mp2 input section with the following keywords:
\begin{description}
\item[econv] \textit{val} energy convergence threshold for the correlation energy. Recommended is 1.e-3 or 1.e-4.
\item[dconv]  \textit{val}  [optional] residual for the convergence of the MP1 wave function. Default is sqrt(econv)*0.1;
\item[freeze]  \textit{val}  number of frozen core electrons. Default is 0.
\item[maxsub]  \textit{val}  size of the subspace of the KAIN solver for the MP1 wave function. Default is 3.
\end{description}
Other options like k, L, or the nuclear correlation factor are taken from the reference input section.
For 6D calculations the polynomial order k must be chosen small to keep calculations reasonably fast, i.e. k=5,6. 
The \texttt{econv} keyword of the 3D reference will be overriden by mp2 to be 100 times tighter than the 6D correlation.
Local orbitals may be used.

\section{Example input}
dft\\
  xc       hf\\
  canon\\
  k        5\\
  econv    1.00e-05\\
  dconv    1.00e-04\\
  maxiter  20\\
  L        20\\
  nuclear\_corrfac  slater  2.0\\
end\\
\\
mp2\\
  econv    1.00e-03\\
  dconv    3.16e-03\\
  maxsub   3\\
  freeze   1\\
end\\
\\
geometry\\
  eprec 1.e-3\\
  O  0.0    0.0 0.0\\
  H  1.4375 0.0 1.15\\
  H -1.4375 0.0 1.15\\
end\\

\bibliographystyle{aip}
\bibliography{references}


\end{document}

